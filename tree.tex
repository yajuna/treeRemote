\documentclass[12pt]{article}

\usepackage{amsmath, amsthm, ulem, graphicx, marvosym, fancyhdr, amscd, amssymb, mathrsfs,color,subcaption,nicefrac,hyperref}
\usepackage{framed}
\usepackage{tikz}
\usetikzlibrary{decorations.pathreplacing,angles,quotes}
\usepackage[inline]{showlabels}
\usepackage{multicol}

\newcommand\Z{{\mathbb Z}}
\newcommand\F{{\mathbb F}}
\newcommand\N{{\mathbb N}}
\newcommand\bE{{\mathbf E}}
\newcommand\p{{\mathscr P}}
\newcommand\R{{\mathbb R}}
\newcommand\Q{{\mathbb Q}}

\tikzset{every label/.style={font=\footnotesize,inner sep=1pt}}
\newcommand{\stencilpt}[4][]{\node[circle,fill,draw,inner sep=1.5pt,label={above left:#4},#1] at (#2) (#3) {}}


\topmargin -.5in
\evensidemargin-.3in
\oddsidemargin-.3in
\textheight 9.5in
\textwidth 7in

\newtheorem*{theorem}{Theorem}
\newtheorem{lemma}{Lemma}
\newtheorem{prop}{Proposition}
\newtheorem{cor}{Corollary}


%\setlength{\voffset}{-1in}

\everymath={\displaystyle}

%%% 
% Framed mini-page environment 
%%%
\newsavebox{\fmbox}
   \newenvironment{ans}
     {\begin{lrbox}{\fmbox}\begin{minipage}{13cm} \textbf{Answer:} \par}
     {\end{minipage}\end{lrbox}\fbox{\usebox{\fmbox}} \vspace{0.25cm}}

\newcommand{\lcm}{\operatorname{lcm}}
\newcommand{\ds}{\displaystyle}

%\pagestyle{fancy}

\renewcommand{\headrulewidth}{0pt}
%\rhead{}
%\lhead{}
%\cfoot{\Huge\Stopsign}

\pagestyle{empty}

\title{Tree Power Notes}


\author{author\thanks{University of Washington, WA} %\and author\thanks{institue2} 
\and author\footnotemark[1]}

\begin{document}

\maketitle

\noindent ABSTRACT. 

\vspace{1cm}

\noindent Key words: 


\section*{Introduction}  \textcolor{red}{Literature review:}



\section{Assumptions}

%With these assumptions, the DE problem reduces to 1D.
\begin{enumerate}
\item We assume all trees we consider are roughly the same. That means same height, same radius
\item With the previous assumption, we place all devices at the same height (therefore eliminating the $z$ variable in the heat equation), and same depth
\end{enumerate}

\section{To do (April)}
\begin{enumerate}
\item get heat programs running in Python3 (eliminate all syntax error)
\begin{itemize}
\item write source terms from Potter 2002
\item write 2D heat code with both second order centered difference, and ADT method.
\end{itemize}
\item find appropriate parameters for testing (ask hardware group)
\item ways to validate computation (numerical analysis methods)
\item write a documentation (i.e., this file??)
\end{enumerate}

\section{PDE model of temperature within a tree stem}
\subsection{Parameters \textcolor{red}{ to collect and verify. Nick}}
Notations: 
\begin{enumerate}
\item $T$: temperature (K)
\item $\rho$: density (kg/m$^3$)
\item $c$: specific heat (J/(kgK))
\item $t$: time (s)
\item $k$: thermal conductivity (W/(mK))
\item $r$: distance from center of the tree (m)
\item $\phi$: azimuth angle, measured clockwise with south being $\phi=0$ (radian/degree), [\ref{heat}] assumes $\partial k/\partial \phi=0$
\item $\alpha$: albedo of the surface
\end{enumerate}

\subsection{The heat equation in cylindrical coordinates}
The standard heat equation for temperature distribution in cylindrical coordinates is 
\begin{equation}
\rho c\frac{\partial T}{\partial t}=\nabla\cdot(k\nabla T)=\frac{1}{r}\frac{\partial}{\partial r}\bigg(kr\frac{\partial T}{\partial r}\bigg)+\frac{1}{r}\frac{\partial}{\partial \phi}\bigg(\frac{k}{r}\frac{\partial T}{\partial \phi}\bigg)\label{heat}
\end{equation}

We add source terms for the diffusion equation to obtain
\begin{equation}
\rho c\frac{\partial T}{\partial t}=\nabla\cdot(k\nabla T)-\frac{1}{\Delta r}[H+(1-\alpha)(S_{dir}+S_{dif})+(IR_{in}-IR_{out})]
\end{equation}

In the following, we explain the source terms: 
\begin{enumerate}
\item 
Free convective heat loss/gain happens when the tree surface temp is different from the ambient temp. Forced convection happens when there is wind. We denote the heat loss/gain by $H$, and 
\begin{equation}
H=h(T_{sfc}-T_{air}),
\end{equation}
with $h$ (W/($m^2$K)) being the convective heat transfer coefficient. Here
\begin{equation}
h=h_{free}-h_{forced},
\end{equation}
more details in \ref{heat}. 

\item $S_{dir}+S_{dif}$ represents direct solar radiation, plus diffusion solar radiation. 

\item $IR_{in}-IR_{out}$ represents wave radiation from and to the tree. More details see [\ref{heat}].

\end{enumerate}

\textcolor{red}{Question to think about: Take part of the source term as boundary condition? What is the boundary condition with source term already implemented?}

In this work, we solve the heat equation to model the temperature in tree trunks, for the purpose of energy harvesting. We modify above source terms to fit the parameters for trees in Washington state.



%\href{https://pycav.readthedocs.io/en/latest/api/pde/crank_nicolson.html}{ref for CN 2D}

\subsection{FD scheme for the heat equation}
We solve the above equation numerically with the Finite-Difference (FD) method, as the FD method is simple to implement, and robust.  

A detailed discussion on axi-symmetric heat equation is in [\ref{2017book}]. We model the heat equation to depend on the azimuth angle $\phi$, as trees in Washington state experience different sunlight and therefore growth in different direction. \textcolor{red}{pg.251, section 3.5.6}

For simplicity of discussion, we assume $\rho,\ c, k$ are all constants. This is a reasonable assumption, as the trees should not have parameters that change drastically within the trunk. Then the constant coefficient equation \eqref{heat} simplifies to 
\begin{equation}
\frac{\rho c}{k}\frac{\partial T}{\partial t}=\frac{\partial^2 T}{\partial r^2}+\frac{1}{r}\frac{\partial T}{\partial r}+\frac{1}{r^2}\frac{\partial^2 T}{\partial \phi^2}\label{simple_heat}
\end{equation}

With centered FD discretization in space, and forward in time for example, we have
\begin{itemize}\begin{multicols}{2}
\item $\frac{\partial T}{\partial t}\approx\frac{T^1_{i,j}-T^0_{i,j}}{\Delta t}$
\item $\frac{\partial T}{\partial r}\approx\frac{T^0_{i+1,j}-T^0_{i-1,j}}{2\Delta r}$
\item $\frac{\partial^2 T}{\partial r^2}\approx\frac{T^0_{i-1,j}+T^0_{i+1,j}-2T^0_{i,j}}{(\Delta r)^2}$
\item $\frac{\partial^2 T}{\partial \phi^2}\approx\frac{T^0_{i,j-1}+T^0_{i,j+1}-2T^0_{i,j}}{(\Delta \phi)^2}$
\end{multicols}\end{itemize}

Here $T^n_{i,j} \approx T(r_i,\phi_j,t_n)$ denotes the numerical solution at the point $(r_i,\phi_j,t_n)$. In general, we compute the heat equation iteratively in time with the general time forwarding scheme: 
\begin{equation}
\frac{T^1_{i,j}-T^0_{i,j}}{\Delta t}=D\ T_{i,j},
\end{equation}
where the right hand side is a discretization in space, with $D$ being the difference operator. Depending on $D$, the above {\it continuous differential equation} can be discretized in the following two ways:
\subsubsection{Forward in time, centered difference Method (FTCD)}
The FTCD is the most straightforward method, where we directly apply the above listed discretization to the heat equation. 
\begin{align}
T^1_{i,j}=& \frac{k\Delta t}{\rho c}\bigg[\underbrace{\frac{1}{(\Delta r)^2}\bigg(T^0_{i-1,j}+T^0_{i+1,j}-2T^0_{i,j}\bigg)}_{\frac{\partial^2T}{\partial r^2}}+\underbrace{\frac{1}{2r_i\Delta r}\bigg(T^0_{i+1,j}-T^0_{i-1,j}\bigg)}_{\frac{1}{r}\frac{\partial T}{\partial r}}\nonumber\\
+& \underbrace{\frac{1}{r_i^2(\Delta \phi)^2}\bigg(T^0_{i,j-1}+T^0_{i,j+1}-2T^0_{i,j}\bigg)}_{\frac{1}{r^2}\frac{\partial^2 T}{\partial \phi^2}}\bigg]+T^0_{i,j},
\end{align}
with $k_1= \frac{k\Delta t}{\rho c(\Delta r)^2}$, $k_2 = \frac{k\Delta t}{2\rho cr_i\Delta r}$, and $k_3= \frac{k\Delta t}{\rho cr_i^2(\Delta \phi)^2}$, above equation can be implemented as
\begin{equation}
T^1_{i,j}= k_1\bigg(T^0_{i-1,j}+T^0_{i+1,j}-2T^0_{i,j}\bigg)+k_2\bigg(T^0_{i+1,j}-T^0_{i-1,j}\bigg)
+k_3\bigg(T^0_{i,j-1}+T^0_{i,j+1}-2T^0_{i,j}\bigg)+T^0_{i,j},
\end{equation}
where we obtain $T^1_{i,j}$ each time step with $T$ values in several locations from previous step. This system of equations can be solved directly. See implementation (that needs modification) in {\tt heatPolar.py}. Here we can denote the general scheme to be 
\begin{equation}
\frac{T^1_{i,j}-T^0_{i,j}}{\Delta t}=D\ T^0_{i,j}.
\end{equation}

\begin{center}\begin{tikzpicture}
  \stencilpt{-2,0}{i-2}{$T^0_{i-1,j}$};
  \stencilpt[blue]{ 0,0}{i}  {$T^0_{i,j}$};
    \stencilpt{ 2,0}{i+2}{$T^0_{i+1,j}$};
  \stencilpt{0,-2}{j-2}{$T^0_{i,j-1}$};
  \stencilpt{0, 2}{j+2}{$T^0_{i,j+1}$};
  \draw (j-2) -- (j+2)
        (i-2) -- (i+2);
\end{tikzpicture}

Space information at each time step.
\end{center}
Imagine one more axis normal to the paper, pointing outwards above the blue dot. That would be $T^1_{i,j}$. 

\subsubsection{Crank-Nicolson Method (CNM)}
We also solve the heat equation with the Crank-Nicolson method, as it is easy to modify given the FTCD method. To put simply, the general scheme is modified by averaging the space information of the current step and the next step:
\begin{equation}
\frac{T^1_{i,j}-T^0_{i,j}}{\Delta t}=\frac{1}{2}D\ T^0_{i,j}+\frac{1}{2}D\ T^1_{i,j}.
\end{equation}
The specific scheme is 
\begin{align}
& T^1_{i,j}-0.5k_1\bigg(T^1_{i-1,j}+T^1_{i+1,j}-2T^1_{i,j}\bigg)-0.5k_2\bigg(T^1_{i+1,j}-T^1_{i-1,j}\bigg)
-0.5k_3\bigg(T^1_{i,j-1}+T^1_{i,j+1}-2T^1_{i,j}\bigg)\nonumber\\
&= 0.5k_1\bigg(T^0_{i-1,j}+T^0_{i+1,j}-2T^0_{i,j}\bigg)+0.5k_2\bigg(T^0_{i+1,j}-T^0_{i-1,j}\bigg)
+0.5k_3\bigg(T^0_{i,j-1}+T^0_{i,j+1}-2T^0_{i,j}\bigg)+T^0_{i,j},
\end{align}
here $k_1,\ k_2,\ k_3$ are previously defined. 



\subsection{Implementation in Python and numerical results}
With the FTCD method, we need to worry about numerical stability. By the von Neumann stability analysis, we write $T(r,\phi, t_n)=e^{ikr+il\phi}$, and $T(r,\phi, t_{n+1})=Ge^{ikr+il\phi}$. We force $G\leq 1$. Plug $T(r,\phi, t_n)=T^0_{i,j}$ and $T(r,\phi, t_{n+1})=T^1_{i,j}$ into the Finite Difference equation, we get (we will temporarily denote $c=\frac{k}{\rho c}$)
\begin{equation}
G = c\Delta t \bigg[\frac{1}{(\Delta r)^2}(e^{-ik\Delta r}+e^{ik\Delta r}-2)+\frac{1}{2r_i\Delta r}(e^{ik\Delta r}-e^{-ik\Delta r})+\frac{1}{(r_i\Delta \phi)^2}(e^{-il\Delta \phi}+e^{il\Delta \phi}-2)\bigg]+1
\end{equation}
As $r_i\geq \Delta r$, we replace all $r_i$ with $\Delta r$, and we get 
\begin{equation}
G\leq c\Delta t \bigg[\frac{1}{(\Delta r)^2}(2\cos(k\Delta r)-2)+\frac{1}{2(\Delta r)^2}2i\sin(k\Delta r)+\frac{1}{(\Delta r\Delta \phi)^2}(2\cos(l\Delta \phi)-2)\bigg]+1
\end{equation}
We request $|G|\leq 1$, and it comes down to the following inequality:
\begin{equation}
\bigg[1-\frac{c\Delta t}{(\Delta r)^2}4\sin^2(k\Delta r/2)-\frac{c\Delta t}{(\Delta r\Delta \phi)^2}4\sin^2(l\Delta \phi/2)\bigg]^2+\bigg[\frac{c\Delta t}{(\Delta r)^2}\sin(k\Delta r)\bigg]^2\leq 1
\end{equation}
Worst case is when $k\Delta r = l\Delta \phi = \pi$, and we get the stability requirement: 
\begin{equation}
\bigg(1-\frac{c\Delta t}{(\Delta r)^2}4-\frac{c\Delta t}{(\Delta r\Delta \phi)^2}4\bigg)^2\leq 1
\end{equation}


For vectorization, we rewrite the above equation as 
\begin{equation}
T^1_{i,j}= (k_1-k_2)T^0_{i-1,j}+(k_1+k_2)T^0_{i+1,j}+(1-2k_1-2k_3)T^0_{i,j}+
+k_3T^0_{i,j-1}+k_3T^0_{i,j+1}
\end{equation}

Read Chapter 9 of Randy LeVeque. 
%\newpage

%\section{To do (Jan 31)}
%\begin{enumerate}
%\item (DISCUSSION ITEM) Understand the hardware more: which equations to use, heat? thermoelectric? both? \textcolor{blue}{DONE. use heat equation. Goal is to model the temp in the tree and in the environment. At a same height. see\ref{sapwood}.}

%\item (ACTION ITEM) Write code (Matlab/Python) to solve equations (equations see next section). \textcolor{red}{ref: http://www.claudiobellei.com/2016/11/10/crank-nicolson/}

%\textcolor{red}{http://www.claudiobellei.com/2016/10/15/explicit-parabolic/}

%\textcolor{blue}{1D code. Keep code in github repo: https://github.com/yajuna/treeRemote}

%\item Verify the assumptions. 
%\begin{enumerate}
%\item With vertical drilling and data collection, find the best height. Call $h_0$\textcolor{blue}{see \ref{sapwood}}
%\item With horizontal drilling and data collection, find the best depth. Call $r_0$
%\end{enumerate}

%With the assumption that we have found the ``best" height and radius (best: highest voltage, most activities, depending on the device??), we simplify the problem into a 1D problem. 


%\item Find appropriate parameters in DE. For now, we take simple ones. \textcolor{red}{ISSUE: if real parameters are small or have varying magnitudes, might cause unexpected numerical errors}\textcolor{blue}{Waiting for confirmation from Nick.}


%\end{enumerate}

%\section{Differential Equations}

%\begin{tikzpicture}
%\def\angle{60}%
%\pgfmathsetlengthmacro{\xoff}{2cm*cos(\angle)}%
%\pgfmathsetlengthmacro{\yoff}{1cm*sin(\angle)}%
%\draw [thick, fill=gray!10] (\xoff,-\yoff) circle[x radius=8cm, y radius=4cm] ++(3*\xoff,-3*\yoff) node{Heat Source};
%\draw [thick, fill=gray!50] (0.5*\xoff,-0.5*\yoff) circle[x radius=5cm, y radius=2.5cm] ++(1.5*\xoff,-1.5*\yoff) node{Thermoelectric Region};
%\draw [thick, fill=gray!80] (0,0) circle[x radius=2cm, y radius=1cm] node{Heat Sink};
%\end{tikzpicture}

%%ref for tikz: https://tex.stackexchange.com/questions/495446/drawing-concentric-ellipses-with-text-with-tikz

%We modify the equations from [\ref{thermoelectric}], to radial equations.
%Heat source and thermoelectric material in Fig.2[\ref{thermoelectric}] is now along the radius $r$. (\textcolor{red}{ASK ABOUT HARDWARE. Fig.1, and Fig.2. from ref})

%Heat sink and resource for region $R$ are governed by the heat equation
%\begin{equation}
%\rho_R\ C_{vR}\ \frac{\partial T}{\partial t}=\nabla\cdot (k_R\nabla T),\label{heat}
%\end{equation}

%here, $T$ is the temperature, $t$ time, $\rho_R$ the density for region $R$, $C_{vR}$ specific heat, $k_R$ is the thermal conductivity. 

%The middle layer is the thermoelectric region, governed by 
%\begin{align}
%\rho_{m}\ C_{vm}\ \frac{\partial T}{\partial t}&=\sigma\bE\cdot\bE-\sigma\cdot \alpha\bE\cdot \nabla T+\nabla\cdot[(k_m+\sigma\alpha^2T)\nabla T-\sigma\alpha T\bE],\label{elec}\\
%\frac{\partial \rho}{\partial t}&=\nabla\cdot (-\sigma\bE+\sigma\alpha\nabla T).\label{chargedensity}
%\end{align}
%Here $\bE$ is the electric field, $\rho$ the charge density, $\sigma$ the electric conductivity, and $\alpha$ the Seebeck coefficient (\textcolor{red}{with temp dependence $\alpha=\alpha(T)$. Need curve fitting to decide}). 

%\textcolor{red}{DO NOT KNOW ANY PARAMETERS}

%\subsection{Reduction to 1D problem}
%Based on our assumptions, we reduce spatial dependence to only on radius $r$: 
%\begin{align}
%\rho_R\ C_{vR}\ \frac{\partial T}{\partial t}&=\frac{\partial }{\partial r}\big(k_R\frac{\partial T}{\partial r}\big),\label{heat1d}\\
%\rho_{m}\ C_{vm}\ \frac{\partial T}{\partial t}&=\sigma E^2-\sigma \alpha E\cdot \frac{\partial T}{\partial r}+\frac{\partial }{\partial r}[(k_m+\sigma\alpha^2T)\frac{\partial T}{\partial r}-\sigma\alpha T E],\label{elec1d}\\
%\epsilon\frac{\partial E}{\partial t}&=J_0-\sigma E+\sigma\alpha\frac{\partial T}{\partial r}.\label{chargedensity1d}
%\end{align}

%Here \eqref{chargedensity1d} is the result of integrating \eqref{chargedensity}, and $J_0$ a constant. 

%\subsection{Radial boundary conditions}

%From outside inwards, the boundary conditions are:
%\begin{itemize}
%\item Outside: tree bark has $T_{amb}$.
%\item Between Heat Source and Thermoelectric region, a voltage $V_0$ is generated, \textcolor{red}{heat flux and temperature??}
%\item \textcolor{red}{section A. in Yan paper}

%Our goal is to write Python scripts that would simulate the temperature distribution in the trunk of the tree and the ambient environment. We will use the heat model for both. See the following reference for more detail:
%\begin{enumerate}
%\item Within a tree stem: see [\ref{heat}]
%\item Ambient temperature: see [\ref{airtempforest}] and [\ref{evergreen}]
%\end{enumerate}

%\end{itemize}

\newpage
Reference\textcolor{red}{BAD FORMAT. FOR convience only}
\begin{enumerate}
%%%%%%%%%%%% A
\item Yan. D. et al, Time-Dependent Finite-Volume Model of Thermoelectric Devices, IEEE, Transactions on industry applications, Vol. 50, No.1, Jan/Feb 2014\label{thermoelectric}

\item MIT notes online\label{mit}

\item Potter, A., Andresen, J., A finite-difference model of temperatures and heat flow within a tree stem, Can. J. For. Res. \textbf{32}, 548-555 (2002)\label{heat}

\item Bownman, W. et al, Sapwood temperature gradients between lower stems and the crown do not influence estimate of stand-level stem CO$_2$ efflux, Tree Physiology, 28, 1553-1559\label{sapwood}

\item Chen, J., et al, An empirical model for predictin diurnal air-temperature gradients from edge into old-growth Douglas-fir forest, Ecological Modeling, Vol. 67, Issues 2-4, pg 179-198, June 1993\label{airtempforest}

\item Tanja, S. et al., Air temperature triggers the recovery of evergreen boreal forest photosynthesis in spring, Global Change Biology, vol 9, issue 10, pg 1410-1426, Oct 2003\label{evergreen}

\item Linge S., Langtangen H.P. (2017) Diffusion Equations. In: Finite Difference Computing with PDEs. Texts in Computational Science and Engineering, vol 16. Springer, Cham\label{2017book}

\end{enumerate}




\end{document}
